\documentclass{article}
\usepackage{graphicx} % Required for inserting images
\usepackage{listings}
\usepackage{float}
\usepackage{color}
\usepackage{booktabs} 
\definecolor{codegreen}{rgb}{0,0.6,0}
\definecolor{codegray}{rgb}{0.5,0.5,0.5}
\definecolor{codepurple}{rgb}{0.58,0,0.82}
\definecolor{backcolour}{rgb}{0.95,0.95,0.92}
\usepackage[numbers]{natbib}
\usepackage{hyperref}

\lstdefinestyle{mystyle}{
    backgroundcolor=\color{backcolour},   
    commentstyle=\color{codegreen},
    keywordstyle=\color{magenta},
    numberstyle=\tiny\color{codegray},
    stringstyle=\color{codepurple},
    basicstyle=\footnotesize,
    breakatwhitespace=false,         
    breaklines=true,                 
    captionpos=b,                    
    keepspaces=true,                 
    numbers=left,                    
    numbersep=5pt,                  
    showspaces=false,                
    showstringspaces=false,
    showtabs=false,                  
    tabsize=2
}

\lstset{style=mystyle}


\begin{abstract}
In recent years, integrating satellite imagery with deep learning (DL) architectures has become an effective approach for environmental monitoring tasks, including forest wildfire detection. Nevertheless, this integration requires substantial high-quality labeled data to train the DL models accurately. Leveraging the capabilities of Google Earth Engine’s Python API, this study introduces a streamlined Python-based methodology to efficiently build, label, process, and evaluate a large-scale bi-temporal high-resolution satellite imagery dataset for DL-driven forest wildfire detection. Known as the California Wildfire GeoImaging Dataset (CWGID), this dataset comprises over 100,000 labeled 'before' and 'after' wildfire image pairs. An analysis of the dataset using pre-trained and adapted Convolutional Neural Network (CNN) architectures, such as VGG16 and EfficientNet, achieved accuracies of respectively 82\% and 93\%. The pipeline described in this paper explains how Python can be used to build a high-resolution satellite imagery dataset and how to train DL models with it, providing an efficient framework for improving environmental monitoring efforts. 
\end{abstract}

\section{Introduction}\label{introduction}
Generally, building a satellite imagery dataset involves a time-intensive approach. Recently, Google Earth Engine (GEE) \citep{gorelick2017google} has revolutionized this process by providing an extensive, cloud-based platform for the efficient collection, processing, and analysis of satellite imagery. Furthermore, GEE’s Python API allows its users to easily query their platform and automatically download cloud-free large-scale satellite imagery datasets from multiple satellite collections, such as Sentinel-2. This technology has facilitated environmental monitoring by providing automatic and timely access to high-quality satellite imagery. 

Satellite imagery-based change detection and forest monitoring have traditionally depended on manually identifying specific features and applying predefined algorithms and models, including differential analysis, thresholding techniques, and clustering and classification algorithms. However, these algorithms and models often fail to capture the full complexity of the studied data. Thus, integrating deep learning (DL) methods with satellite imagery offers a more dynamic and precise approach, one capable of handling the patterns and variability associated with imagery data. 

Python facilitates the use of DL in environmental monitoring by providing a rich ecosystem of libraries and tools, such as TensorFlow \citep{tensorflow2015-whitepaper}, which contains multiple existing DL architectures that can be adapted and used with satellite imagery. Nevertheless, the integration of DL and Remote Sensing images requires multiple processing steps, such as having smaller imagery tiles and adapting the models to use GeoTIFF data, among others. 

Furthermore, these DL algorithms require a substantial amount of labeled data to effectively learn and identify change \citep{Alzubaidi2021ReviewOD}. Therefore, the development of labeled high-resolution satellite imagery datasets is important and relevant for addressing environmental problems. Currently, the availability of labeled high-quality satellite imagery datasets is an obstacle to developing DL models for environmental change detection \citep{Adegun2023}. 

This paper presents a methodology, entirely implemented in Python, to streamline the creation and evaluation, via DL, of satellite imagery datasets. The methodology covers the entire workflow: from data acquisition, labelling, and preprocessing to model adaptation, training, and evaluation. Specifically, this approach is applied to create and validate a high-resolution dataset for forest wildfire detection, the California Wildfire GeoImaging Dataset (CWGID).
\section{Building a Sentinel-2 Satellite Imagery Dataset}
To construct the CWGID, a multi-step process is needed.
\subsection{Gathering and Refining Historic Wildfire Polygon Data from California}
The initial step is to gather geo-referenced forest wildfire polygon data from California, sourced from the Fire and Resource Assessment Program (FRAP) maintained by the California Department of Forestry and Fire Protection \citep{california_department_of_forestry_and_fire_protection_2024}. This FRAP data includes perimeters of past wildfires and serves as the geographic reference needed to select satellite imagery with GEE. 

Then, in Python, the Pandas library \citep{pandas1} is used to organize the forest wildfire attribute data into a Pandas DataFrame, which is then filtered to align with the launch date and operational phase of the Sentinel-2 satellites, selected for their open-source, high-resolution imagery capabilities \citep{DRUSCH201225}. Additionally, the dates are adjusted to fall within the green-up period, avoiding the winter and fall seasons where snow cover could interfere with identifying burnt areas.

Next, the data is formatted to meet GEE’s querying specifications:
\begin{itemize}
    \item A 15-day range for pre- and post-fire dates is generated and added to the DataFrame.
    \item Using Pandas, the date ranges are formatted to meet GEE's requirements.
    \item Using the pyproj library \citep{pyproj2023}, the recorded point coordinates are converted from NAD83 to WGS84 to facilitate the querying process.
    \item With the geopy \citep{geopy} library, the coordinates of the squared region of interest are calculated, featuring a side length of 15 miles.
\end{itemize}


\subsection{Downloading the Imagery Data Using GEE's Python API}
GEE is a cloud-based platform for global environmental data analysis. It combines an extensive archive of satellite imagery and geospatial datasets with powerful computational resources to enable researchers to detect and quantify changes on the Earth’s surface. GEE’s Python API offers an accessible interface for automating the process of satellite imagery downloads, making it a popular tool for environmental monitoring and research projects.

Multiple steps are needed to set up the GEE's Python API. First, a project is created in Google Cloud Console and the Earth Engine API is enabled. Authentication and Google Drive editing rights are configured to effectively manage and store the downloaded imagery. Following the setup, the Earth Engine Python API is installed on a local machine, and the necessary authentications are performed to initialize the API. 

Then, a Python script is developed to automate the download of images depicting the pre- or post-wildfire data using GEE's Python API (see Code \ref{download}). To download three-channel RGB GeoTIFF imagery, the bands B4, B3, and B2 need to be specified (different band compositions can be selected in this step).  

More specifically, the script is configured with a for loop to iterate through each entry in the DataFrame, extracting necessary parameters such as date ranges, region of interest (ROI) coordinates, and center coordinates of each wildfire polygon. Also, the script is designed to specify parameters such as the desired image collection and a threshold for cloud coverage. Thus, tiles exhibiting more than 10\%  cloud coverage are automatically excluded to maintain data quality.

Finally, the images are downloaded and exported to Google Drive in a GeoTIFF format. 

\begin{lstlisting}[language=Python, label= download, caption = Downloading Images Using GEE and a DataFrame]
#Authenticate into EE
ee.Authenticate()

#Initialize EE
ee.Initialize()

# Initialize a list for iteration
num = []
for i in range(0, x):  # modify depending on your number of data points
    num.append(i)

# Define the bands you want to download
bands_rgb = ['B4', 'B3', 'B2']  # Red, Green, Blue

event_type = 'before'  # Toggle this to 'after' as needed

for i in num:
    # Set the center point
    point = ee.Geometry.Point(round(data[i][1], 2), round(data[i][2], 2))

    # Call an image for the selected point
    tile = ee.ImageCollection('COPERNICUS/S2') \
        .filterBounds(point) \
        .filterDate(data[i][3], data[i][4]) \
        .sort('CLOUDY_PIXEL_PERCENTAGE') \
        .first()

    # Check the properties of the image
    image_properties = tile.getInfo()
    cloudy_percentage = image_properties.get(
        'properties', {}).get('CLOUDY_PIXEL_PERCENTAGE', 0)

    if cloudy_percentage <= 10:
        # Define the ROI
        roi = ee.Geometry.Polygon([[
            [data[i][9], data[i][10]],
            [data[i][11], data[i][12]],
            [data[i][13], data[i][14]],
            [data[i][15], data[i][16]],
            [data[i][17], data[i][18]]]])

        # Select RGB bands
        rgb_tile = tile.select(bands_rgb)

        # Export RGB image to Google Drive
        rgb_task = ee.batch.Export.image.toDrive(**{
            'image': rgb_tile,
            'description': f'RGB_{event_type}Fire'
            + str(data[i][0]),
            'folder': f'GEE_FireImagesRGB_{event_type}',
            'scale': 10,  # Adjust the scale as needed
            'region': roi.getInfo()['coordinates'],
            'crs': 'EPSG:4326',
            'fileFormat': 'GeoTIFF',
        })
        rgb_task.start()

        # Wait for the task to complete
        while not rgb_task.status()['state'] in \
                ['COMPLETED', 'FAILED', 'CANCELLED']:
            print(f'RGB Task for {event_type} fire is',
                  rgb_task.status()['state'])
            # before checking again

        # Check the final task status
        rgb_task_status = rgb_task.status()
        print(f"RGB Task Status for {event_type} fire:",
              rgb_task_status)
        print(f"RGB Task Error Message for {event_type}
              fire: ", rgb_task_status.get(
            "error_message", "No error message"))
    else:
        print(
            f"Skipping image {i} due to cloudy percentage
            ({cloudy_percentage} %) > 10 %")

\end{lstlisting}
\subsection{Creating the Ground Truth Wildfire Labels}
Ground truth masks are essential in forest wildfire detection and general land cover classifications \citep{8113128}. In this project, masks are generated to label the data.

To do this, Python is used to rasterize the combined geometry of the forest wildfire polygon data and the downloaded post-wildfire RGB satellite imagery. Specifically, the forest wildfire polygons are accessed in Python using GeoPandas \citep{geopandas} and reprojected to match the coordinate system of the satellite imagery (EPSG:4326). Then, each post-wildfire RGB image is locally and temporarily downloaded from Google Drive, with essential properties such as width, height, transform, and bounds extracted using the rasterio library \citep{rasterio}. Next, the geometry column from the forest wildfire polygon data is extracted and intersected with each image bound using Python's shapely \citep{shapely} library. Finally, binary masks are created by rasterizing the combined geometries. These masks match the dimensions of the satellite images, ensuring that each pixel labeled as wildfire damage corresponds directly to the polygon data. The binary masks are saved temporarily in GeoTIFF format and are uploaded to a dedicated Google Drive folder (see Code \ref{lst:fire_polygons_processing}).  All the temporary local files were deleted to clear space and maintain system efficiency.  
\begin{lstlisting}[language=Python, label= lst:fire_polygons_processing, caption = Building Ground Truth Masks]
import geopandas as gpd
import numpy as np
import rasterio
from shapely.geometry import box, shape
import os

# Define the path to your Shapefile - replace with your specific path
shapefile_path = "YourShapefileDirectory/FirePolygons.shp"

# Read the Shapefile using geopandas
gdf = gpd.read_file(shapefile_path)

# Reproject the shapefile to EPSG:4326 to match the satellite imagery coordinate system
gdf = gdf.to_crs(epsg=4326)

# Create a directory to store the output raster masks if it doesn't already exist
output_dir = "raster_masks"
if not os.path.exists(output_dir):
    os.makedirs(output_dir)

# Iterate through the rows in the attribute table
for index, row in gdf.iterrows():
    object_id = row["OBJECTID"]
    image_path = f"YourImageDirectory/RGB_AfterFire{object_id}.tif"
    # Check if the image exists
    if os.path.exists(image_path):
        # Open the image using rasterio
        with rasterio.open(image_path) as src:
            image_width = src.width
            image_height = src.height
            image_transform = src.transform
            image_bounds = box(
                src.bounds[0], src.bounds[1],
                src.bounds[2], src.bounds[3]
            )

        # Extract the geometry column
        geom = shape(row["geometry"])
        clipped_geom = geom.intersection(image_bounds.envelope)

        if not clipped_geom.is_empty:
            # Create a two-dimensional label by rasterizing the
            # clipped geometry
            clipped_mask = rasterio.features.geometry_mask(
                [clipped_geom],
                out_shape=(image_height, image_width),
                transform=image_transform,
                invert=True,
            )

            # Save the image with the two-dimensional label overlay
            output_image_path
            = f"{output_dir}/Masked_{object_id}.tif"
            with rasterio.open(
                output_image_path,
                "w",
                driver="GTiff",
                width=image_width,
                height=image_height,
                count=1,
                dtype=np.uint8,
                crs=src.crs,
                transform=image_transform,
            ) as dst:
                dst.write(clipped_mask.astype(np.uint8), 1)
\end{lstlisting}

In the final ground truth masks, the pixel values are set to zero if they are outside of a wildfire polygon, indicating unaffected areas, and set to one if they are within the polygon boundaries, indicating areas affected by a forest wildfire.
\subsection{Image Segmentation and Data Preparation for Deep Learning Architectures}
Next, the satellite images and their corresponding ground-truth masks are cropped into smaller tiles that maintain the imagery's spatial resolution (10m). Often, satellite imagery needs to be resized and downscaled to accommodate deep learning (DL) architectures, which can result in the loss of essential details such as subtle indicators of early-stage wildfires. Moreover, using smaller images enhances the efficiency of DL models by lowering computational demands and speeding up training times \citep{hu2015transferring,marmanis2016deep}. 

To do this, a tile size of 256x256 pixels is specified and each RGB image is downloaded individually from Google Drive to a temporary local folder. Using Python’s rasterio library, the original RGB images are opened to obtain their dimensions. Then, the number of rows and columns for the tiles is calculated based on the chosen tile size. Next, a rasterio Window object is used to extract the corresponding portion from the original image and read the RGB data, ensuring the order of the bands (B4, B3, and B2).

The segmented RGB tiles are then saved as GeoTIFF files using the tifffile Python library \citep{tifffile}. This is a critical step to maintain the integrity of the three-channel RGB data, as the rasterio library alone can alter the color of the images during saving. Additionally, the metadata of the saved tiles is updated to include georeferencing information and to modify parameters such as width, height, and transform (see Code \ref{lst:save_rgb_tiles}).

A similar approach is used to segment the binary masks, specifying that the images contain only one band. 
\begin{lstlisting}[language=Python, caption={Function to Crop RGB Image Tiles}, label= lst:save_rgb_tiles]
from rasterio import Window
from tifffile import imwrite

# Function to save image tiles without changing the data type
def save_rgb_tiles(image_path, output_folder, tile_size, parent_name):
    with rasterio.open(image_path) as src:
        height = src.height
        width = src.width

        num_rows = height // tile_size
        num_cols = width // tile_size

        tile_counter = 1  # Initialize the tile counter

        for i in range(num_rows):
            for j in range(num_cols):
                window = Window(j * tile_size, i * tile_size, tile_size, tile_size)

                # Read the original data without modifications
                # Assuming band order B4, B3, B2
                tile = src.read((1, 2, 3), window=window)
                tile_name = f"{parent_name}_tile_{tile_counter}.tif"
                tile_path = os.path.join(output_folder, tile_name)

                # Save the tile using tifffile without changing
                # data type
                imwrite(tile_path, tile)

                # Update metadata with georeferencing information
                meta = src.meta.copy()
                transform = src.window_transform(window)
                meta.update({
                    'width': tile_size,
                    'height': tile_size,
                    'transform': transform
                })

                with rasterio.open(tile_path, 'w', **meta) as dst:
                    dst.write(tile)
                tile_counter += 1  # Increment the tile counter
\end{lstlisting}

By combining the capabilities of rasterio for efficient geospatial data handling and the tifffile library for preserving the RGB data during saving, the original images are cropped into smaller RGB tiles. This approach preserves the resolution and the georeferencing information of the images, preparing them to train DL applications.
\subsection{Data Augmentation for Wildfire Damage Detection}
It is essential to have a balanced dataset that includes both forest wildfire-affected and unaffected areas. Initially, the dataset comprised 82,082 tiles, but only 4,847 of these showed signs of wildfire damage, indicating a significant class imbalance. This could lead to overfitting, with the model biased towards undamaged landscapes.

To address this imbalance and enhance the model's accuracy in detecting wildfire-affected areas, data augmentation techniques are implemented. Specifically, functions using the rasterio and NumPy \citep{numpy} libraries are developed to perform image transformations, which include rotating the GeoTIFF tiles by 90°, 180°, and 270°, and flipping them horizontally and vertically. These transformations are applied to the imagery and the mask tiles that contain fire polygons and to their corresponding pre-wildfire RGB GeoTIFF tiles.

Finally, the augmentation process increased the diversity of the training data and resulted in a total of 111,093 pairs of labeled RGB GeoTIFF image tiles. The final dataset consists of 82,011 negative instances (no wildfire damage) and 29,082 positive instances (wildfire damage), significantly improving the class balance and reducing the risk of overfitting.
\section{Evaluating the CWGID Using Convolutional Neural Networks}
The CWGID is tested for demonstration purposes using the well-known CNN architectures VGG16 and EfficientNet, both implemented using Python's TensorFlow’s Keras API.

\subsection{VGG16 Implementation}
VGG16 \citep{Simonyan15} is a deep CNN designed for image processing and classification tasks. It consists of 13 convolutional layers, 5 pooling layers, and 3 fully connected layers. This model is adapted to train on the dataset and detect positive and negative instances of forest wildfires. The VGG16 architecture only trains on the post-wildfire 3-channel RGB imagery, which contains both.

To use this CNN architecture with GeoTIFF satellite images, which typically store geographic data not supported by most deep-learning libraries, a custom function to feed the data to the model is built using the rasterio library and the shuffle function from the Scikit-learn library \citep{sklearn1,sklearn2} (see Code \ref{lst:custom_generator}). 

The data labeling is established by specifying the base paths to the training and testing directories for damaged and undamaged classes. The function reads batches of 32 GeoTIFF files, shuffles them, and processes them into a three-dimensional array compatible with VGG16. Since the VGG16 model needs to discern the presence or the absence of wildfire damage in the image, the ground-truth masks are not directly used. They are only used to define the imagery file paths.
Thus, the labeling of this model is based on the presence of "Damaged" (value of 1) in the file paths. All the other files are labeled as "Undamaged" (value of 0), which simplifies the binary classification task.
\begin{lstlisting}[language=Python, caption={Custom Function to Feed GeoTIFF Files to the VGG16 Model}, label= lst:custom_generator]
from sklearn.utils import shuffle

# Define the base paths for training and testing
base_training_path = "Insert your training file path"
base_testing_path = "Insert your testing file path"

def custom_image_generator(file_paths, batch_size):
    while True:
        file_paths = shuffle(file_paths)

        for i in range(0, len(file_paths), batch_size):
            batch_files = file_paths[i : i + batch_size]
            images, labels = [], []

            for file in batch_files:
                with rasterio.open(file) as src:
                    image = src.read()
                    # Channels last
                    image = np.moveaxis(image, 0, -1)

                label = 1 if "/Damaged/" in file else 0
                images.append(image)
                labels.append(label)

            yield np.array(images), np.array(labels)
\end{lstlisting}

To detect wildfire-affected areas with VGG16, the model is initiated using the pre-trained weights from the ImageNet dataset. The convolutional base is frozen to preserve the integrity of the learned features and to focus on the training of the added layers. A Flatten operation is applied to the output to transform the two-dimensional feature maps into a one-dimensional vector. Then, a fully connected dense layer with 1024 neurons and ReLU activation is added. Finally, the network becomes a single-neuron dense layer with a sigmoid activation function that produces a probability score of wildfire damage. The complete model is compiled with the Adam optimizer \citep{kingma2014adam}, binary cross-entropy loss \citep{goodfellow2016deep}, and accuracy, precision, and recall as the performance metrics (see Code \ref{lst:vgg16_wildfire_detection}). The Adam optimizer is chosen because it adapts the learning rate during training. In the context of satellite image classification, where the landscape can vary significantly across images, Adam’s ability to adjust the learning rate is important. Also, the binary cross-entropy loss is chosen because it is tailored for binary classification tasks, such as classifying between damaged and undamaged areas in satellite images.
\begin{lstlisting}[language=Python, caption={Adaptation of VGG16 for Wildfire Damage Detection}, label= lst:vgg16_wildfire_detection]
from keras.models import Model
from keras.layers import Flatten, Dense
from keras.optimizers import Adam
from keras.applications import VGG16
from keras.metrics import Precision, Recall

# Define the base model using pre-trained ImageNet weights
base_model = VGG16(weights='imagenet', include_top=False, input_shape=(256, 256, 3))

# Freeze the convolutional base to prevent weights from being updated
for layer in base_model.layers:
    layer.trainable = False

# Add custom layers on top of the base model
x = Flatten()(base_model.output) # Flatten the output to make it suitable for dense layers
x = Dense(1024, activation='relu')(x) # Add a dense layer with 1024 units and ReLU activation
output = Dense(1, activation='sigmoid')(x) # Output layer with a sigmoid activation for binary classification

# Construct the complete model
model = Model(inputs=base_model.input, outputs=output)

# Compile the model with Adam optimizer and binary cross-entropy loss
model.compile(optimizer=Adam(learning_rate=0.0001),
              loss='binary_crossentropy',
              metrics=['accuracy', Precision(), Recall()])

\end{lstlisting}

The VGG16 model is trained and validated on 95594 three-channel RGB images (45\% of the data, divided as follows: 80\% for training and
10\% for validation) and is tested on 10621 images (5\% of the data). Training took 1953 minutes for 10 epochs. 
\subsection{EfficientNet Implementation}
\begin{figure}[H]
    \centering
    \includegraphics[width=0.9\textwidth]{6channel.png}
    \caption{Representation of a 6 Channel RGB GeoTIFF Input. A: Representation of a 3-channel RGB GeoTIFF forested area \textit{before} a wildfire B: Visual example of a 3-channel RGB GeoTIFF forested area \textit{after} a wildfire.}. \label{fig1}
\end{figure}
EfficientNet \citep{tan2019} is a CNN architecture that uniformly scales network width, depth, and resolution with a fixed set of scaling coefficients. EfficientNet’s architecture begins with a base model, EfficientNet-B0, designed to find the optimal baseline network configuration. The following versions of the network are further scaled versions of B0, offering multiple models for different computational budgets. 

For this project, Efficient-B0 is adapted, trained, and tested using pre- and post-wildfire RGB GeoTIFF imagery pairs from the CWGID. To do this, an approach commonly known as Early Fusion (EF) is employed, where six-channel GeoTIFF files that combine the image pairs into a single input are built (see Figure \ref{fig1}). To use this, a custom function that allows the use of 6-channel GeoTIFF data with Efficient-B0 is built with the rasterio library and Keras' Sequence function (see Code \ref{lst:six_channel_generator}). 
\begin{lstlisting}[language=Python, caption={Custom Function to Feed Multi-Channel GeoTIFF Files to the EfficientB0 Model}, label= lst:six_channel_generator]
from keras.utils import Sequence
# Custom Data Generator for Six-Channel Images


class SixChannelGenerator(Sequence):
    # Change this depending on your own data
    def __init__(
        self,
        file_paths,
        labels,
        batch_size=32,
        dim=(256, 256),
        n_channels=6,
        shuffle=True,
    ):
        self.file_paths = file_paths
        self.labels = labels
        self.batch_size = batch_size
        self.dim = dim
        self.n_channels = n_channels
        self.shuffle = shuffle
        self.on_epoch_end()

    def __len__(self):
        return int(np.ceil(len(self.file_paths) / self.batch_size))

    def __getitem__(self, index):
        batch_paths = self.file_paths[
            index * self.batch_size: (index + 1) * self.batch_size
        ]
        batch_labels = self.labels[
            index * self.batch_size: (index + 1) * self.batch_size
        ]
        batch_x = np.empty(
            (len(batch_paths), *self.dim,
             self.n_channels), dtype=np.float32
        )
        batch_y = np.array(batch_labels, dtype=np.float32)

        for i, path in enumerate(batch_paths):
            with rasterio.open(path) as src:
                img = src.read()[: self.n_channels,
                                 : self.dim[0],
                                 : self.dim[1]]
                # Convert from channels_first to
                # channels_last format
                img = np.moveaxis(img, 0, -1)
                batch_x[i,] = img / 255.0  # Normalize images

        return batch_x, batch_y

    def on_epoch_end(self):
        if self.shuffle:
            temp = list(zip(self.file_paths, self.labels))
            np.random.shuffle(temp)
            self.file_paths, self.labels = zip(*temp)
\end{lstlisting}

Furthermore, the data labeling and its use are formulated using the same pipeline as VGG16, by specifying the base paths to the training and testing directories for both damaged and undamaged classes. As above, the labeling of this model is based on the presence or the absence of ’Damaged’ in the imagery file paths.

The model is built with an input tensor defined to accept 256px by 256px images with six channels. A standard convolutional layer is applied to this input tensor to perform an initial convolution operation. The base EfficientNet-B0 model is loaded both without the top layers (classification layers) and the pre-trained weights, as these were trained using 3-channel images. A Dropout layer is added, with a specified rate of 30\%, to reduce overfitting. A dense layer is also added with 1024 neurons and ReLU activation (see Code \ref{lst:efficientnet_training}). In the end, the network outputs a probability indicating the likelihood of the image showing ’damaged’ versus ’undamaged’ areas.
\begin{lstlisting}[language=Python, caption={Initializing and Training EfficientNetB0 for Six-Channel Image Input}, label= lst:efficientnet_training]
from keras.layers import Input, Conv2D, Dense, GlobalAveragePooling2D, Dropout
from keras.models import Model
from keras.applications import EfficientNetB0
from keras.callbacks import EarlyStopping, ModelCheckpoint, ReduceLROnPlateau

# Model Adaptation for Six-Channel Input


# Change this depending on your own data
def create_efficientnet_six_channel(input_shape=(256, 256, 6), dropout_rate=0.3):
    input_tensor = Input(shape=input_shape)
    x = Conv2D(3, (3, 3), padding='same', activation='relu')(input_tensor)
    base_model = EfficientNetB0(
        include_top=False, input_tensor=x, weights=None)
    x = GlobalAveragePooling2D()(base_model.output)
    x = Dropout(dropout_rate)(x)
    x = Dense(1024, activation='relu')(x)
    output = Dense(1, activation='sigmoid')(x)
    model = Model(inputs=input_tensor, outputs=output)
    return model


# Initialize the model
model = create_efficientnet_six_channel()
# Model Training with Callbacks for Optimal Training Control
early_stopping = EarlyStopping(
    monitor='val_loss', patience=5, restore_best_weights=True)
model_checkpoint = ModelCheckpoint('yourmodel.keras', save_best_only=True)
reduce_lr = ReduceLROnPlateau(
    monitor='val_loss', factor=0.2, patience=2, min_lr=1e-6, verbose=1)

# Compile the model
model.compile(optimizer='adam',
              loss='binary_crossentropy',
              metrics=['accuracy', Precision(), Recall()])
\end{lstlisting}

Before training, class weights are calculated using the Scikit-learn library. Furthermore, Keras callbacks are set up for early stopping, model checkpointing, and reducing the learning rate when the validation loss plateaus, which helps optimize the training process. As VGG16, the model is compiled with the Adam optimizer and binary cross-entropy loss, with accuracy, precision, and recall as performance metrics.

For testing purposes, this EF architecture is trained on 23833 pre- and post-wildfire image pairs (22.5\% of the data, divided as follows: 80\% for training and 10\% for validation) and is tested on 2716 pre- and post-wildfire image pairs (2.5\% of the data). Training took 933 minutes for 13 epochs. 

\subsection{DL Results}
The results in Table \ref{tab:1} compare the performance of the two neural network architectures used on the CWGID. Each architecture's performance metrics are assessed, including loss, accuracy, precision, recall, and training time. Also, the percentage of the CWGID used to train each architecture is specified. 

Both models are trained with subsets of the CWGID dataset, with the 6-channel input EF EfficientNet model using 25\% of the data and the VGG16 model using 50\% of the data. 

The 6-channel input EfficientNet achieved the lowest loss (0.178) and highest accuracy (92.6\%), making it the most effective model in terms of overall performance. Its precision was at 92.5\% and its recall was 81.9\%, indicating a strong ability to identify damaged areas correctly with some missed detections. VGG16, despite processing a larger dataset, had a higher loss (1.220) but maintained an accuracy of 83.2\%. Its precision was 74.9\% and its recall was 58.8\%. This architecture took a longer training time. The EfficientNet model shows a relatively faster training time when considering the epochs and the amount of data used.  The 6-channel input EfficientNet demonstrated superior performance in accurately classifying areas affected by wildfires with high precision and accuracy, making it an efficient choice for applications where quick and accurate assessments are critical. Moreover, it is important to mention that VGG16 was tested with Six-Channel inputs but proved computationally intensive.

 \begin{table}[H]
    \centering
    \caption{Performance Metrics of the Models. \label{tab:1}}
    \begin{tabular}{ccc}
    \toprule
    \textbf{Model} & \textbf{6-channel input EfficientNet} & \textbf{VGG16} \\
    \midrule
    Data from the CWGID (\%) & 25 & 50 \\
    Loss & 0.178 & 1.220 \\
    Accuracy & 0.926 & 0.832 \\
    Precision & 0.925 & 0.749 \\
    Recall & 0.819 & 0.588 \\
    Time (minutes) & 933 & 1953 \\
    \bottomrule
    \end{tabular}
\end{table}
\section{Discussion}

The methodology developed in this study facilitates the creation of large, reliable, and cloud-free bi-temporal satellite imagery datasets, which are essential for accurate environmental monitoring. Also, by leveraging historical and curated datasets for the labeling, our approach maximizes the value of existing data, accurately labels the imagery, and minimizes the need for manual labeling.

Furthermore, the methodology was successfully evaluated, as the CWGID enabled the adapted CNN models to learn and identify the signs of wildfire damage with high accuracy, specifically with the EF EfficientNet-B0 model. Additionally, the proposed methodology produced a greater number of examples compared to other similar datasets in the field \citep{al-dabbagh2023uni,SEYDI2022108999,Hunan}, which can boost the performance of deep learning models, making them more effective at detecting and analyzing environmental changes. Moreover, the high precision and recall metrics for EfficientNet, particularly the recall of 81.9\%, display the model’s capacity to correctly identify wildfire-affected areas with fewer false negatives, which remains critical for rapid and effective wildfire monitoring.

On top of that, the dataset's bi-temporal collection of pre- and post-wildfire satellite imagery also offers the possibility to use different and accurate approaches for detecting forest wildfires with satellite imagery. The comparative analysis of VGG16 and EfficientNet shows the advantage of employing a bitemporal approach to satellite imagery analysis under a single input. By evaluating \textit{before} and \textit{after} images as coupled inputs, the model identifies changes that a single post-event imagery might not reveal, as shown by VGG16. 


\section{Conclusion}

This paper illustrates a Python methodology that integrates DL with satellite imagery to enhance environmental monitoring. Specifically, the development of the California Wildfire GeoImaging Dataset (CWGID) and its evaluation using DL architectures are outlined. The approach explained throughout this work improves the accurate and timely detection of forest wildfires and can be applied to a broad range of environmental issues, beyond just wildfire detection.
Firstly, historical data, Python-based tools, and Google Earth Engine are leveraged to build and label a large, cloud-free Sentinel-2 satellite imagery dataset depicting forest wildfire examples. Then, this dataset is used to train well-known CNN architectures, enabling the effective detection of forest wildfires within the satellite imagery. An accuracy of 92.6 \% with the EfficientNet model was obtained, indicating the potential effectiveness of the methodology.
Future studies will include using Fully Convolutional Networks (FCNs), like U-Nets \citep{DBLP:RonnebergerFB15}, because they allow pixel-wise classification. It is anticipated that these networks will enhance the accuracy of detecting and mapping wildfire-affected areas, leveraging the detailed ground truth masks included within the CWGID.


\bibliography{mybib}
\bibliographystyle{unsrtnat}

